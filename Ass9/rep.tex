%% LyX 2.1.4 created this file.  For more info, see http://www.lyx.org/.
%% Do not edit unless you really know what you are doing.
\documentclass[11pt,a4paper,english]{article}
\usepackage{mathptmx}
\usepackage[T1]{fontenc}
\usepackage[utf8]{inputenc}

\makeatletter

%%%%%%%%%%%%%%%%%%%%%%%%%%%%%% LyX specific LaTeX commands.
\pdfpageheight\paperheight
\pdfpagewidth\paperwidth


%%%%%%%%%%%%%%%%%%%%%%%%%%%%%% User specified LaTeX commands.
\title{Report for assignment 1}
\author{Your Name (Roll Number)}

\makeatother

\usepackage{babel}
\begin{document}

\title{Assignment 9}


\author{Avijit Ghosh}


\author{14CH3FP18}


\date{20 March 2016}
\maketitle
\begin{abstract}
Simulation of reading large chunks of data using Fibonacci Heaps
\end{abstract}

\part{Introduction}

Fibonacci heaps are interesting the!oretically because they have asymptotically
good runtime guarantees for many operations. In particular, \textbf{insert,
peek, and decrease-key} all run in amortized $O(1)$ time. \textbf{dequeueMin}
and \textbf{delete} each run in amortized  $O(lgn)$ time. This allows
algorithms that rely heavily on \textbf{decrease-key}  to gain significant
performance boosts. For example, Dijkstra's algorithm  for single-source
shortest paths can be shown to run in $O(m+nlgn)$ using  a Fibonacci
heap, compared to $O(mlgn)$ using a standard binary or binomial  heap.
  Internally, a Fibonacci heap is represented as a circular, doubly-linked
 list of trees obeying the min-heap property. Each node stores pointers
 to its parent (if any) and some arbitrary child. Additionally, every
 node stores its degree (the number of children it has) and whether
it  is a \textquotedbl{}marked\textquotedbl{} node. Finally, each
Fibonacci heap stores a pointer to  the tree with the minimum value.

To insert a node into a Fibonacci heap, a singleton tree is created
and  merged into the rest of the trees. The \textbf{merge} operation
works by simply  splicing together the doubly-linked lists of the
two trees, then updating  the min pointer to be the smaller of the
minima of the two heaps. Peeking  at the smallest element can therefore
be accomplished by just looking at  the min element. All of these
operations complete in $O(1)$ time.   

The tricky operations are \textbf{dequeueMin} and \textbf{decreaseKey}.
\textbf{dequeueMin} works  by removing the root of the tree containing
the smallest element, then  merging its children with the topmost
roots. Then, the roots are scanned  and merged so that there is only
one tree of each degree in the root list.  This works by maintaining
a dynamic array of trees, each initially null,  pointing to the roots
of trees of each dimension. The list is then scanned  and this array
is populated. Whenever a conflict is discovered, the  appropriate
trees are merged together until no more conflicts exist. The  resulting
trees are then put into the root list. A clever analysis using  the
potential method can be used to show that the amortized cost of this
 operation is $O(lgn)$. 

The other hard operation is \textbf{decreaseKey}, which works as follows.
First, we  update the key of the node to be the new value. If this
leaves the node  smaller than its parent, we're done. Otherwise, we
cut the node from its  parent, add it as a root, and then mark its
parent. If the parent was  already marked, we cut that node as well,
recursively mark its parent,  and continue this process. This can
be shown to run in $O(1)$ amortized time  using yet another clever
potential function. Finally, given this function,  we can implement
\textbf{delete} by decreasing a key to $-\infty$, then calling  \textbf{dequeueMin}
to extract it.


\part{Heap Operations}


\section{Inserting a node}

The following procedure inserts node x into Fibonacci heap H, assuming
of course that the node has already been allocated and that key{[}x{]}
has already been filled in.

FIB-HEAP-INSERT(H, x)

1 degree{[}x{]} 0

2 p{[}x{]} NIL

3 child{[}x{]} NIL

4 left{[}x{]} x

5 right{[}x{]} x

6 mark{[}x{]} FALSE

7 concatenate the root list containing x with root list H 

8 if min{[}H{]} = NIL or key{[}x{]} < key{[}min{[}H{]}{]}

9 then min{[}H{]} x

10 n{[}H{]} n{[}H{]} + 1

To determine the amortized cost of FIB-HEAP-INSERT, let H be the input
Fibonacci heap and H' be the resulting Fibonacci heap. Then, t(H')
= t(H) + 1 and m(H') = m(H), and the increase in potential is

((t(H) + 1) + 2m(H)) - (t(H) + 2m(H)) = 1 . Since the actual cost
is O(1), the amortized cost is O(1) + 1 = O(1).


\section{Finding the minimum node}

The minimum node of a Fibonacci heap H is given by the pointer min{[}H{]},
so we can find the minimum node in O(1) actual time. Because the potential
of H does not change, the amortized cost of this operation is equal
to its O(1) actual cost.


\section{Uniting two Fibonacci heaps}

The following procedure unites Fibonacci heaps H1 and H2, destroying
H1 and H2 in the process.

FIB-HEAP-UNION(H1,H2)

1 H MAKE-FIB-HEAP()

2 min{[}H{]} min{[}H1{]}

3 concatenate the root list of H2 with the root list of H

4 if (min{[}H1{]} = NIL) or (min{[}H2{]} NIL and min{[}H2{]} < min{[}H1{]})

5 then min{[}H{]} min{[}H2{]}

6 n{[}H{]} n{[}H1{]} + n{[}H2{]}

7 free the objects H1 and H2

8 return H Lines 1-3 concatenate the root lists of H1 and H2 into
a new root list H. Lines 2, 4, and 5 set the minimum node of H, and
line 6 sets n{[}H{]} to the total number of nodes. The Fibonacci heap
objects H1 and H2 are freed in line 7, and line 8 returns the resulting
Fibonacci heap H. As in the FIB-HEAP-INSERT procedure, no consolidation
of trees occurs.

The change in potential is

(H) - ((H1) + (H2))

= (t(H) + 2m(H)) - ((t(H1) + 2 m(H1)) + (t(H2) + 2m(H2)))

= 0, because t(H) = t(H1) + t(H2) and m(H) = m(H1) + m(H2). The amortized
cost of FIB-HEAP-UNION is therefore equal to its O(1) actual cost.


\section{Extracting the minimum node}

The process of extracting the minimum node is the most complicated
of the operations presented in this section. It is also where the
delayed work of consolidating trees in the root list finally occurs.
The following pseudocode extracts the minimum node. The code assumes
for convenience that when a node is removed from a linked list, pointers
remaining in the list are updated, but pointers in the extracted node
are left unchanged. It also uses the auxiliary procedure MELD, which
will be presented shortly.

FIB-HEAP-EXTRACT-MIN(H)

1 z min{[}H{]}

2 if z NIL

3 then for each child x of z

4 do add x to the root list of H

5 p{[}x{]} NIL

6 remove z from the root list of H

7 if z = right{[}z{]}

8 then min{[}H{]} NIL

9 else min{[}H{]} right{[}z{]}

10 MELD(H)

11 n{[}H{]} n{[}H{]} - 1

12 return z

Consolidating the root list consists of repeatedly executing the following
steps until every root in the root list has a distinct degree value.

1. Find two roots x and y in the root list with the same degree, where
key{[}x{]} key{[}y{]}.

2. Link y to x: remove y from the root list, and make y a child of
x. This operation is performed by the FIB-HEAP-LINK procedure. The
field degree{[}x{]} is incremented, and the mark on y, if any, is
cleared.

The procedure MELD uses an auxiliary array A{[}0 . . D(n{[}H{]}){]};
if A{[}i{]} = y, then y is currently a root with degree{[}y{]} = i.

MELD(H)

1 for i 0 to D(n{[}H{]})

2 do A{[}i{]} NIL

3 for each node w in the root list of H

4 do x w

5 d degree{[}x{]}

6 while A{[}d{]} NIL

7 do y A{[}d{]}

8 if key{[}x{]} > key{[}y{]}

9 then exchange x y

10 FIB-HEAP-LINK(H,y,x)

11 A{[}d{]} NIL

12 d d + 1

13 A{[}d{]} x

14 min{[}H{]} NIL

15 for i 0 to D(n{[}H{]})

16 do if A{[}i{]} NIL

17 then add A{[}i{]} to the root list of H

18 if min{[}H{]} = NIL or key{[}A{[}i{]}{]} < key{[}min{[}H{]}{]}

19 then min{[}H{]} A{[}i{]}

FIB-HEAP-LINK(H, y, x)

1 remove y from the root list of H

2 make y a child of x, incrementing degree{[}x{]}

3 mark{[}y{]} FALSE 

The actual cost of extracting the minimum node can be accounted for
as follows. An O(D(n)) contribution comes from there being at most
D(n) children of the minimum node that are processed in FIB-HEAP-EXTRACT-MIN
and from the work in lines 1-2 and 14-19 of Meld. It remains to analyze
the contribution from the for loop of lines 3-13. The size of the
root list upon calling Meld is at most D(n) + t(H) - 1, since it consists
of the original t(H) root-list nodes, minus the extracted root node,
plus the children of the extracted node, which number at most D(n).
Every time through the while loop of lines 6-12, one of the roots
is linked to another, and thus the total amount of work performed
in the for loop is at most proportional to D(n) + t(H). Thus, the
total actual work is O(D(n) + t(H)).

The potential before extracting the minimum node is t(H) + 2m(H),
and the potential afterward is at most (D(n) + 1) + 2m(H), since at
most D(n) + 1 roots remain and no nodes become marked during the operation.
The amortized cost is thus at most

O(D(n) + t(H)) + ((D(n) + 1) + 2m(H)) - (t(H) + 2m(H))

= O(D(n)) + O(t(H)) - t(H)

= O(D(n)), since we can scale up the units of potential to dominate
the constant hidden in O(t(H)). Intuitively, the cost of performing
each link is paid for by the reduction in potential due to the link
reducing the number of roots by one.
\end{document}
