\documentstyle[11pt]{article}
\textheight 9.25 in
\topmargin  -0.65 in
\footheight 0.4 in
\textwidth 6.5in
\oddsidemargin -0.2in
\evensidemargin -0.4in
\renewcommand{\textfraction}{0.05}
\renewcommand{\topfraction}{0.9}
\renewcommand{\dbltopfraction}{0.9}
\renewcommand{\bottomfraction}{0.9}
\renewcommand{\floatpagefraction}{0.9}
\renewcommand{\dblfloatpagefraction}{0.9}
\setcounter{topnumber}{3}
\setcounter{bottomnumber}{3}
\begin{document}
\baselineskip=12pt
\parskip = 5pt
\pagenumbering{arabic}
\pagestyle{empty}

\noindent
CS/EE 548 \hfill Handout \#2 \\
Fall 96-97 \hfill \\
Myers\hfill \\

\centerline{\Large\bf HOMEWORK \#1: Fundamental Algorithms}

\vspace*{0.4in}

\noindent
This homework is due in class on Thursday, October 10, 1996. \\
Be sure to show ALL steps to get full credit. 

\begin{enumerate}

\item {\bf Shortest/longest path} (20 points) \\
Do part (a) on Figure 1 and parts (b) and (c) on Figure 2:
\begin{itemize}
\item[(a)] Compute shortest-paths from vertex 0 to each vertex using 
           the Bellman-Ford algorithm (Algorithm 2.4.2). 
\item[(b)] Compute longest-paths from vertex 0 to each vertex using
           a modified Bellman-Ford algorithm (i.e., change min to max 
           and initialize to $-\infty$ instead of $\infty$).
\item[(c)] Compute longest-paths from vertex 0 to each vertex using
           the Liao-Wong algorithm (Algorithm 2.4.3).
\end{itemize}		

\item {\bf Vertex cover} (20 points) \\ 
Do the following for the graph in Figure 3:
\begin{itemize}
\item[(a)] Compute the vertex cover using the Vertex\_Cover\_V algorithm 
           (Algorithm 2.4.4).
\begin{itemize}
\item[(i)] Selecting vertex 2 first.
\item[(ii)] Selecting vertex 3 first.
\end{itemize}		
\item[(b)] Compute the vertex cover using the Vertex\_Cover\_E algorithm 
           (Algorithm 2.4.5).
\end{itemize}		

\item {\bf Graph coloring} (20 points) \\ 
Do the following for the set of intervals below: \\
$(0,3), (2,4), (7,12), (6,15), (5,9), (1,7), (11,13)$
\begin{itemize}
\item[(a)] Construct an interval graph labelling the vertices in the 
           order given above.
\item[(b)] Color the graph using the Vertex\_Color algorithm 
           (Algorithm 2.4.6).
\item[(c)] Color the graph using the Left Edge algorithm
           (Algorithm 2.4.7).
\end{itemize}		

\item {\bf Clique covering and partitioning} (20 points) \\ 
Find a clique partition for the graph in Figure 4 using 
Clique\_Partition and Max\_Clique algorithms (Algorithms 2.4.8 and 2.4.9).

\item {\bf Satisfiability and cover} (20 points) \\ 
Do the following for the hypergraph in Figure 5: 
\begin{itemize}
\item[(a)] Represent minimum edge-covering problem as a boolean satisfiability
           problem.
\item[(b)] Construct the (vertex/edge) incidence matrix $A_I$.
\item[(c)] Solve the covering problem using the Exact\_Cover
           algorithm (Algorithm 2.5.4).
\end{itemize}		

\item {\bf EXTRA CREDIT:} (10 points) \\ 
Problem 2.8.3 from the book.

\end{enumerate}
	
\end{document}

