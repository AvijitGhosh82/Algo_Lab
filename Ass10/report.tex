%% LyX 2.1.4 created this file.  For more info, see http://www.lyx.org/.
%% Do not edit unless you really know what you are doing.
\documentclass[english]{article}
\usepackage[T1]{fontenc}
\usepackage[latin9]{inputenc}

\makeatletter
%%%%%%%%%%%%%%%%%%%%%%%%%%%%%% User specified LaTeX commands.
\usepackage{babel}

\makeatother

\usepackage{babel}
\begin{document}

\title{Assignment 10}


\author{Avijit Ghosh}


\author{14CH3FP18}


\date{25.03.2016}

\maketitle
 
\begin{abstract}
Finding arbitrage opportunities in currency conversion using the Bellman-Ford
Algorithm 
\end{abstract}

\part{Introduction}

Arbitrage is the use of discrepancies in currency exchange rates to
transform one unit of a currency into more than one unit of the same
currency. Suppose, 1 U.S. dollar bought 0.82 Euro, Euro bought 129.7
Japanese Yen,1 Japanese Yen bought 12 Turkish Lira, and one Turkish
Lira bought 0.0008 U.S. dollars. Then, by converting currencies, a
trader can start with U.S. dollar and buy 1.02 U.S. dollars, thus
turning a \$0.02 profit. Suppose that we are given n currencies nxn
table of exchange rates, we should find if such a cyclic arbitrage
opportunity exists.


\part{Bellman Ford Algorithm}

function BellmanFord(list vertices, list edges, vertex source)

init distance{[}{]},predecessor{[}{]}

\indent // This implementation takes in a graph, represented as //
lists of vertices and edges, and fills two arrays (distance and predecessor)
with shortest-path (less cost/distance/metric) information

\indent // Step 1: initialize graph for each vertex v in vertices:

\indent distance{[}v{]}= inf

\indent // At the beginning , all vertices have a weight of infinity

\indent predecessor{[}v{]}= null

\indent // And a null predecessor

\indent \indent distance{[}source{]} = 0

\indent \indent // Except for the Source, where the Weight is zero

\indent//Step 2: relax edges repeatedly for i from 1 to size(vertices)-1:

for each edge (u, v) with weight w in edges:

\indent if distance{[}u{]} + w < distance{[}v{]}:

\indent \indent distance{[}v{]} = distance{[}u{]} + w

\indent \indent predecessor{[}v{]} = u

\indent // Step 3: check for negative-weight cycles

\indent for each edge (u, v) with weight w in edges:

\indent\indent if distance{[}u{]} + w < distance{[}v{]}:

\indent\indent\indent error \textquotedbl{}Graph contains a negative-weight
cycle\textquotedbl{}

\indent\indent return distance{[}{]}, predecessor{[}{]}


\part{Using this in our Arbitrage Problem}

A solution to the problem is achieved when the product of subsequents
weights is >1. This can be converted to a BellMan ford problem by
converting the edged weights to negative logs and then finding negative
cycles in the graph.


\part{Complexity Analysis}

\medskip{}


Time complexity: We first modify each weight $w$ by $-log(w)$ in
time $O(|E|)$. We then use Bellman Ford to find cycles in time $O(|V||E|)$.
Therefore our asymptotic time complexity is $O(|V||E|)$. Printing
takes time $O(V)$

\medskip{}


Proof of correctness: Assume for proof by contradiction we have a
risk-free currency exchange opportunity in the graph but it is not
a cycle. We note that Bellman Ford, by iterating through all edges
in finding shortest paths, will detect all graphs in a cycle. Our
exchange opportunity must be present in the graph so we can follow
a currency's edges to find this opportunity. If we follow paths leading
from a source currency (which we know is included in the path of an
opportunity) we must either: 
\begin{itemize}
\item i) either eventually find a path back to this currency
\item ii) never return to this currency
\end{itemize}
\smallskip{}


In case ii) we never return to our currency. This means we have not
found a risk-free opportunity because this implies we return to a
currency with positive value.

\smallskip{}


In case i) we consider first the subcase where we return with a positive
path (or value 0).

\smallskip{}


In this subcase we have lost money returning to our currency (or made
no money in case the path has sum of weights 0). This means it is
not a risk-free opportunity because there is no gain in exercising
it.

\smallskip{}


In the case i) subccase where we have a negative path: This means
we have found a path returning to our original currency that makes
the user money. Clearly (given an unchanging graph), we can exercise
this path multiple times. This means it is a negative cycle and will
be reported by Bellman Ford.

\smallskip{}


As we have shown how all cases except the last subcase are contradictory,
we have proven that returning negative cycles found by Bellman Ford
will return and only return risk-free currency exchanges.
\end{document}
