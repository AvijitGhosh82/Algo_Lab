\documentclass{article}
\usepackage[utf8]{inputenc}

\title{Assignment 3}
\author{Avijit Ghosh }
\date{January 2016}

\begin{document}

\maketitle

\section{The Skyline Problem:}
\textbf{\large{}1. Rectangular Buildings}{\large \par}


a. The series of buildings is generated randomly (format: left, height,
right)

b. They are arranged using Quicksort ( O(nlogn))

c. A structure called Skyline is used here which is basically a linked
list with two data points (left, height), in each node.

d. Now the sorted building array is divided into left and right parts
until the size is 1 (divide)

e. The base case is when there is only one building, in this case
the skyline data structure to be returned will have only two nodes:
(left, height)-->(right, 0)

f. The left and right lists(skylines) are now merged recursively on
the basis of their x coordinates (left) in increasing order .(conquer).

g. In this merge function two things are kept in mind: If the incoming
node has the same x coordinate as the previous, but has a different
height, the bigger height is taken. And if the incoing node has same
height but a different x coordinate, it is dropped, as it is redundant
(building overlapped by another). 

h. Thus the above algorithm is a simple merge sort type algorithm
which has O(nlogn) complexity.

\textbf{\large{}2. Slant Buildings}{\large \par}


This problem is largely similar to the previous one.

a. Buildings are represented by (left, left height, right, right height)
format.

b. They are arranged using Quicksort ( O(nlogn))

c. A structure called Skyline is used here which is basically a linked
list with two data points (left, height), in each node.

d. Now the sorted building array is divided into left and right parts
until the size is \textbf{\large{}2} (divide). 

e. This is because for two overlapping buildings, there will now be
a third point to be taken into consideration, which is basically the
point of intersection of the two roof lines of the two buildings.

f. 5 points are returned in base case. (2 for left, 2 for right and
1 intersection if valid).

g. The left and right lists(skylines) are now merged recursively on
the basis of their x coordinates (left) in increasing order .(conquer).

h. In this merge function two things are kept in mind: If the incoming
node has the same x coordinate as the previous, but has a different
height, the bigger height is taken. And if the incoing node has same
height but a different x coordinate, it is dropped, as it is redundant
(building overlapped by another). 

i. Thus the above algorithm is a simple merge sort type algorithm
which has O(nlogn) complexity.

\end{document}
